\chapter*{Załącznik A: Instrukcja użycia biblioteki}
\addcontentsline{toc}{chapter}{Załącznik A: Instrukcja użycia biblioteki}
W załączniku opisana została procedura użycia modułu URM na podstawie utworzenia przykładowego projektu i skonfigurowania go do potrzeb biblioteki. Na potrzeby przykładu użyty zostanie program Visual Studio 2022 oraz system budowy MSBuild. Do instalacji pakietów posłuży program vcpkg od firmy Microsoft. 

Biblioteka może zostać zaimportowana w dwóch trybach:
\begin{enumerate}
	\item \textbf{Prekompilowana} - forma dystrybucji, w której moduł przyjmuje formę gotowych do użycia plików .lib, zawierających skompilowane funkcje biblioteki. Dzięki takiemu podejściu można uniknąć każdorazowego rekompilowania kodu przy tworzeniu projektu, a także potencjalnych problemów z kompatybilnością między różnymi wersjami kompilatora. 
	\item \textbf{Kod źródłowy} - bezpośrednie dodanie projektów wraz z kodem źródłowym jako referencji do aplikacji klienta. Dzięki takiemu podejściu możliwa jest dowolna modyfikacja kodu biblioteki, ale kosztem konieczności jej ręcznego skompilowania oraz trudniejszej aktualizacji.
\end{enumerate}

\section*{Wspólna konfiguracja projektu}
Niezależnie od wybranej wersji wymaganym od projektu jest ustawienie kilku kluczowych opcji konfiguracyjnych. 

Ze względu na wykorzystanie zaawansowanych konceptów jak \textit{std::variant} i \textit{concepts}, moduł do swojego działania wymaga języka C++ w wersji 20 lub nowszej. Dodatkowo do poprawnej kompilacji wymagana jest też flaga \textit{/utf-8}, która wymusza na kompilatorze użycie kodowania UTF-8. Na koniec do poprawnego działania z WinAPI koniecznym jest też przestawienie użytego subsystemu na Windows.

Poniżej znajduje się podsumowanie wymaganych ustawień:
\begin{center}
	\begin{tabular}{ |l r|}
		\hline
		\textbf{Ustawienie} & \textbf{Wartość} \\
		\hline
		General -> \textbf{C++ language standard}  & \textbf{>= C++20} \\
		C/C++ -> Command Line -> \textbf{Additional Options} & \textbf{/utf-8} \\
		Linker -> System -> \textbf{SubSystem} & \textbf{Windows} \\
		\hline
	\end{tabular}
\end{center}

\section*{Tryb prekompilowany}
