\chapter{Decyzje projektowe, wybór API}

% % Definicja kolorów dla listingu kodu
% \definecolor{codegreen}{rgb}{0,0.6,0}
% \definecolor{codegray}{rgb}{0.5,0.5,0.5}
% \definecolor{codepurple}{rgb}{0.58,0,0.82}
% \definecolor{backcolour}{rgb}{0.95,0.95,0.92}
% \definecolor{keywordblue}{rgb}{0,0,1}

% % Definicja stylu C++ dla pakietu listings
% % Używamy mathescape=true, aby oryginalne komendy LaTeX w kodzie (np. \emph) były renderowane
% \lstdefinestyle{cppstyle}{
% 	language=C++,
% 	backgroundcolor=\color{backcolour},
% 	commentstyle=\color{codegreen},
% 	keywordstyle=\color{magenta},
% 	numberstyle=\tiny\color{codegray},
% 	stringstyle=\color{codepurple},
% 	basicstyle=\ttfamily\footnotesize,
% 	breakatwhitespace=false,
% 	breaklines=true,
% 	captionpos=b,
% 	keepspaces=true,
% 	numbers=left,
% 	numbersep=5pt,
% 	showspaces=false,
% 	showstringspaces=false,
% 	showtabs=false,
% 	tabsize=2,
% 	mathescape=true, % Pozwala na używanie $...$ do wstawiania komend LaTeX w kodzie
% 	morekeywords={*,ComPtr,hwnd,message,wParam,lParam,uint,LRESULT,CALLBACK,HWND,UINT,IDirect3D9,Direct3DCreate9,D3D_SDK_VERSION,D3DPRESENT_PARAMETERS,Windowed,SwapEffect,D3DSWAPEFFECT_DISCARD,hDeviceWindow,BackBufferFormat,D3DFMT_UNKNOWN,EnableAutoDepthStencil,AutoDepthStencilFormat,D3DFMT_D16,IDirect3DDevice9,CreateDevice,D3DADAPTER_DEFAULT,D3DDEVTYPE_HAL,D3DCREATE_SOFTWARE_VERTEXPROCESSING,D3DCREATE_HARDWARE_VERTEXPROCESSING,D3DDEVTYPE_REF,E_FAIL,HRESULT,FAILED,SUCCEEDED,S_OK,Vertex,DWORD,D3DFVF_CUSTOMVERTEX,D3DFVF_XYZRHW,D3DFVF_DIFFUSE,IDirect3DVertexBuffer9,CreateVertexBuffer,D3DUSAGE_WRITEONLY,D3DPOOL_DEFAULT,Lock,memcpy,Unlock,SetRenderState,D3DRS_CULLMODE,D3DCULL_NONE,D3DRS_LIGHTING,FALSE,D3DRS_ZENABLE,TRUE,Clear,D3DCLEAR_TARGET,D3DCLEAR_ZBUFFER,D3DCOLOR_XRGB,BeginScene,SetFVF,SetStreamSource,sizeof,DrawPrimitive,D3DPT_TRIANGLELIST,EndScene,Present,D3DERR_DEVICELOST,D3DERR_DEVICENOTRESET,TestCooperativeLevel,Reset,ID3D11Device,ID3D11DeviceContext,IDXGISwapChain,ID3D11RenderTargetView,ID3D11DepthStencilView,ID3D11Texture2D,depthStencilBuffer,ID3D11VertexShader,ID3D11PixelShader,ID3D11InputLayout,ID3D11Buffer,windowWidth,windowHeight,DXGI_SWAP_CHAIN_DESC,BufferCount,BufferDesc,Width,Height,Format,DXGI_FORMAT_R8G8B8A8_UNORM,RefreshRate,Numerator,Denominator,BufferUsage,DXGI_USAGE_RENDER_TARGET_OUTPUT,OutputWindow,SampleDesc,Count,Quality,D3D_FEATURE_LEVEL,D3D_FEATURE_LEVEL_11_0,featureLevel,D3D11CreateDeviceAndSwapChain,D3D_DRIVER_TYPE_HARDWARE,D3D_DRIVER_TYPE_WARP,D3D11_CREATE_DEVICE_DEBUG,D3D11_SDK_VERSION,GetBuffer,__uuidof,LPVOID,CreateRenderTargetView,D3D11_TEXTURE2D_DESC,MipLevels,ArraySize,DXGI_FORMAT_D24_UNORM_S8_UINT,D3D11_USAGE_DEFAULT,BindFlags,D3D11_BIND_DEPTH_STENCIL,CPUAccessFlags,MiscFlags,CreateTexture2D,CreateDepthStencilView,OMSetRenderTargets,GetAddressOf,D3DCompile,strlen,nullptr,ID3DBlob,errorBlob,vs_5_0,ps_5_0,D3DCOMPILE_ENABLE_STRICTNESS,GetBufferPointer,GetBufferSize,CreateVertexShader,CreatePixelShader,D3D11_INPUT_ELEMENT_DESC,POSITION,COLOR,DXGI_FORMAT_R32G32B32_FLOAT,DXGI_FORMAT_R32G32B32A32_FLOAT,D3D11_INPUT_PER_VERTEX_DATA,D3D11_APPEND_ALIGNED_ELEMENT,ARRAYSIZE,CreateInputLayout,D3D11_BUFFER_DESC,Usage,D3D11_USAGE_IMMUTABLE,D3D11_BIND_VERTEX_BUFFER,D3D11_SUBRESOURCE_DATA,pSysMem,CreateBuffer,D3D11_VIEWPORT,MinDepth,MaxDepth,TopLeftX,TopLeftY,RSSetViewports,ClearRenderTargetView,ClearDepthStencilView,D3D11_CLEAR_DEPTH,D3D11_CLEAR_STENCIL,IASetVertexBuffers,stride,offset,IASetInputLayout,IASetPrimitiveTopology,D3D11_PRIMITIVE_TOPOLOGY_TRIANGLELIST,VSSetShader,PSSetShader,Draw,DXGI_ERROR_DEVICE_REMOVED,DXGI_ERROR_DEVICE_RESET,OnDeviceLost,ID3D12Device,IDXGIFactory6,ID3D12CommandQueue,IDXGISwapChain3,ID3D12DescriptorHeap,rtvDescriptorSize,ID3D12Resource,ID3D12CommandAllocator,ID3D12RootSignature,ID3D12PipelineState,ID3D12GraphicsCommandList,D3D12_VERTEX_BUFFER_VIEW,ID3D12Fence,UINT64,HANDLE,FrameCount,fenceValues,fenceEvent,exception,std,ThrowIfFailed,DXGI_CREATE_FACTORY_DEBUG,CreateDXGIFactory2,IDXGIAdapter1,GetHardwareAdapter,D3D12CreateDevice,D3D12_COMMAND_QUEUE_DESC,Type,D3D12_COMMAND_LIST_TYPE_DIRECT,Flags,D3D12_COMMAND_QUEUE_FLAG_NONE,CreateCommandQueue,DXGI_SWAP_CHAIN_DESC1,SwapEffect,DXGI_SWAP_EFFECT_FLIP_DISCARD,IDXGISwapChain1,CreateSwapChainForHwnd,MakeWindowAssociation,DXGI_MWA_NO_ALT_ENTER,As,GetCurrentBackBufferIndex,D3D12_DESCRIPTOR_HEAP_DESC,NumDescriptors,D3D12_DESCRIPTOR_HEAP_TYPE_RTV,D3D12_DESCRIPTOR_HEAP_FLAG_NONE,CreateDescriptorHeap,GetDescriptorHandleIncrementSize,D3D12_CPU_DESCRIPTOR_HANDLE,rtvHandle,GetCPUDescriptorHandleForHeapStart,ptr,CreateCommandAllocator,D3D12_ROOT_SIGNATURE_DESC,Init,D3D12_ROOT_SIGNATURE_FLAG_ALLOW_INPUT_ASSEMBLER_INPUT_LAYOUT,D3D_ROOT_SIGNATURE_VERSION_1,D3D12SerializeRootSignature,CreateRootSignature,D3D12_INPUT_ELEMENT_DESC,D3D12_GRAPHICS_PIPELINE_STATE_DESC,InputLayout,pRootSignature,VS,PS,CD3DX12_SHADER_BYTECODE,CD3DX12_RASTERIZER_DESC,CD3DX12_BLEND_DESC,CD3DX12_DEPTH_STENCIL_DESC,PrimitiveTopologyType,D3D12_PRIMITIVE_TOPOLOGY_TYPE_TRIANGLE,NumRenderTargets,RTVFormats,DSVFormat,DXGI_FORMAT_D32_FLOAT,SampleMask,UINT_MAX,CreateGraphicsPipelineState,CreateCommandList,Close,vertexBufferSize,D3D12_HEAP_PROPERTIES,CD3DX12_HEAP_PROPERTIES,D3D12_HEAP_TYPE_DEFAULT,D3D12_RESOURCE_DESC,CD3DX12_RESOURCE_DESC,Buffer,CreateCommittedResource,D3D12_HEAP_FLAG_NONE,D3D12_RESOURCE_STATE_COPY_DEST,D3D12_HEAP_TYPE_UPLOAD,D3D12_RESOURCE_STATE_GENERIC_READ,Map,UINT8,pVertexDataBegin,reinterpret_cast,CD3DX12_RANGE,Unmap,Reset,CopyBufferRegion,D3D12_RESOURCE_BARRIER,CD3DX12_RESOURCE_BARRIER,Transition,D3D12_RESOURCE_STATE_VERTEX_AND_CONSTANT_BUFFER,ResourceBarrier,ExecuteCommandLists,_countof,ppCommandLists,BufferLocation,GetGPUVirtualAddress,StrideInBytes,SizeInBytes,CreateFence,D3D12_FENCE_FLAG_NONE,CreateEvent,GetLastError,HRESULT_FROM_WIN32,WaitForGpu,PopulateCommandList,SetPipelineState,SetGraphicsRootSignature,SetDescriptorHeaps,CD3DX12_VIEWPORT,static_cast,CD3DX12_RECT,RSSetViewports,RSSetScissorRects,D3D12_RESOURCE_STATE_PRESENT,D3D12_RESOURCE_STATE_RENDER_TARGET,CD3DX12_CPU_DESCRIPTOR_HANDLE,OMSetRenderTargets,ClearRenderTargetView,IASetPrimitiveTopology,D3D_PRIMITIVE_TOPOLOGY_TRIANGLELIST,IASetVertexBuffers,DrawInstanced,Signal,GetCompletedValue,SetEventOnCompletion,WaitForSingleObjectEx,previousFrameIndex,CleanupD3D12,CloseHandle,stbi_load,stbi_image_free,stbi_failure_reason,vector,unsigned,char,data,resize,size_t,cerr,cout,wcout,wcerr,endl,hex,WICTextureLoader,CreateWICTextureFromFile,textureResource,textureView,GetDesc,Reset,EffectFactory,Model,CreateFromCMO,CreateFromSDKMESH,wcsstr,meshes,CommonStates,Draw,CXMMATRIX,Assimp,Importer,ReadFile,aiProcess_Triangulate,aiProcess_GenSmoothNormals,aiProcess_FlipUVs,aiProcess_JoinIdenticalVertices,aiProcess_CalcTangentSpace,aiProcess_OptimizeMeshes,aiProcess_OptimizeGraph,aiScene,mFlags,AI_SCENE_FLAGS_INCOMPLETE,mRootNode,GetErrorString,processNode,aiNode,mMeshes,mNumMeshes,push_back,processMesh,aiMesh,mNumChildren,mChildren,MeshData,materialIndex,name,C_Str,vertices,indices,reserve,mNumVertices,XMFLOAT3,XMFLOAT2,XMFLOAT4,mVertices,HasNormals,mNormals,mTextureCoords,mNumFaces,aiFace,mFaces,mNumIndices,substr,find_last_of,directory,processMaterials,ModelData,renderModelAssimp,DrawIndexed,Vector3,Matrix,Quaternion,XMVECTOR,XMMATRIX,SimpleMath}
% }

% % Definicja stylu C# dla pakietu listings
% % Zmieniono tylko na C++, zgodnie z poleceniem, ale zachowujemy oryginalną treść
% \lstdefinestyle{csharpstyle}{
% 	language=C++, % Używamy C++ zgodnie z poleceniem, nawet dla kodu C#
% 	backgroundcolor=\color{backcolour},
% 	commentstyle=\color{codegreen},
% 	keywordstyle=\color{keywordblue}, % Niebieski dla odróżnienia
% 	numberstyle=\tiny\color{codegray},
% 	stringstyle=\color{codepurple},
% 	basicstyle=\ttfamily\footnotesize,
% 	breakatwhitespace=false,
% 	breaklines=true,
% 	captionpos=b,
% 	keepspaces=true,
% 	numbers=left,
% 	numbersep=5pt,
% 	showspaces=false,
% 	showstringspaces=false,
% 	showtabs=false,
% 	tabsize=2,
% 	mathescape=true, % Pozwala na używanie $...$ do wstawiania komend LaTeX w kodzie
% 	morekeywords={Device,VertexBuffer,Control,PresentParameters,SwapEffect,CreateFlags,CustomVertex,PositionColored,GraphicsStream,LockFlags,Usage,Pool,ClearFlags,System,Drawing,Color,PrimitiveType,Microsoft,DirectX,Direct3D,using,new,typeof,WriteOnly,Format,Default,Write,var,float,int,void,class,struct,enum,namespace,public,private,protected,static,readonly,get,set,true,false,null,return,if,else,for,foreach,while,do,switch,case,default,break,continue,try,catch,finally,throw,Form,ClientSize,Text,Show,DirectXException,MessageBox,Disposed,TestCooperativeLevel,DeviceLostException,Console,WriteLine,ToArgb,PaintEventHandler,PaintEventArgs,object,sender,Size,DepthFormat,D16,RenderState,CullMode,Cull,Lighting,ZBufferEnable,DeviceNotResetException}
% }

% \lstset{style=cppstyle} % Ustaw domyślny styl na C++

% % Definicja \real (jeśli nie jest zdefiniowana gdzie indziej)
% \providecommand{\real}[1]{#1}

Przed przystąpieniem do pisania kodu konieczne jest zbadanie dostępnych
na rynku API, które wejdą w skład projektowanego pakietu. Dla
opisywanego programu koniecznym jest między innymi dobranie oraz
przetestowanie różnych wersji API Direct3D, wybór języka programowania,
bibliotek służących do zaawansowanych obliczeń matematycznych, wejścia z
klawiatury i kontrolera, a także bibliotek pomocniczych do otwierania
okna renderowania. Decyzja powinna być oparta o testy wydajności,
wsparcie platformy docelowej, dostępną ilość informacji, jak i o bilans
możliwości do złożoności API.

% Usunięto błędną strukturę enumerate->enumerate
\section{API obsługi okna}

Zanim zostaną podjęte jakiekolwiek decyzje na temat rysowania grafiki,
należy mieć otwarte oraz skonfigurowane okno w taki sposób, aby było na
czym rysować. Testy tego aspektu polegały na stworzeniu minimalnego kodu otwierającego
okno zdolne do rysowania grafiki z wykorzystaniem DirectX.

\subsection{SDL2}

Skrót od \emph{Simple DirectMedia Layer}. Jej drugie wydanie, napisane w
języku C, to wieloplatformowa biblioteka abstrakcji nad niskopoziomowymi
funkcjami API do obsługi audio, klawiatury, myszki, kontrolerów, czy
akceleracji sprzętowej opartej o OpenGL, Direct3D, Metal, bądź Vulkan
\cite{sdl:wiki:2024}. Oficjalnie wspiera systemy Windows, macOS, Linux, iOS oraz
Android \cite{sdl:wiki:2024}. 

Przy pomocy subsystemu okien SDL jest możliwe utworzenie okna, które
następnie można wykorzystać jako powierzchnię rysowania dla zapytań
Direct3D. Przykładowy sposób opisany pseudokodem został przedstawiony
poniżej:

\begin{lstlisting}[caption={Pseudokod inicjalizacji okna SDL2 (oryginalna treść)}, label={lst:sdl2_init}]
	SDL_Init(); // Inicjalizacja biblioteki SDL
	SDL_CreateWindow(); // Utworzenie okna
	SDL_GetWindowWMInfo(); // Pobranie informacji o natywnym oknie
	hwnd = info.info.win.window; // Pobranie natywnego HWND do okna.
	
	while (true) { // Główna pętla wykonywania programu
		// Wyjście z pętli jeśli użytkownik zamknie okno
		if (SDL_PollEvent() == SDL_QUIT)
			break;
	
		draw(); // Kod rysujący
	}
		
	SDL_DestroyWindow(); // Zamknięcie otwartego okna
	SDL_Quit(); // Zamknięcie aplikacji
\end{lstlisting}

SDL jest biblioteką obszerną, ciężką, zawierającą bardzo dużo
niepotrzebnych dla opisywanego projektu elementów. W związku z tym
zastosowana została inna alternatywa.

\subsection{WinAPI}

DirectX został stworzony z myślą o systemie Windows i tylko na nim jest
oficjalnie wspierany. W związku z tym użycie WinAPI, czyli zbioru API
dla systemu Windows, jest najbliższym natywnemu sposobem otwarcia okna
zdolnego do obsługi DirectX. Pozwala także na najwięcej możliwości
związanych z bezpośrednim zarządzaniem oknem oraz zasobami specyficznymi
dla systemu Windows. Wykorzystany sposób otwarcia okna przy pomocy WinAPI został opisany
poniżej przy pomocy analogicznego do poprzedniego pseudokodu.

\begin{lstlisting}[caption={Pseudokod inicjalizacji okna WinAPI (oryginalna treść)}, label={lst:winapi_init}]
	RegisterClassEx(); // Zarejestrowanie klasy obiektu
	CreateWindowEx(); // Utworzenie nowego okna
	ShowWindow(); // Pokazanie nowego okna
	
	while(True) {
		PeekMessage(); // Pobieranie następnej wiadomości
		TranslateMessage(); // Przetwarzanie wiadomości
		DispatchMessage(); // Wysłanie wiadomości do handlera okna
	
		if (message == WM_QUIT)
			break;
		
		draw(); // Kod rysujący
	}
\end{lstlisting}

Do poprawnego działania wymagane jest także stworzenie funkcji handlera
dla okna \emph{WndProc}, która implementuje obsługę różnego rodzaju
sygnałów, które okno może otrzymać. Do podstawowego działania
obejmującego obsługę zamykania okna z poziomu przycisku
\emph{\textbf{x}} można użyć następującej funkcji:

\begin{lstlisting}[caption={Pseudokod funkcji WndProc dla WinAPI (oryginalna treść)}, label={lst:winapi_wndproc}]
	void WndProc (hwnd, message, wParam, lParam) {
		if (message == WM_DESTROY) {
			// Wysłanie zapytania o zamknięcie okna
			// oraz zwolnienie jego zasobów		
			PostQuitMessage();	
		}
		else {
			// Domyślna funkcja obsługi okna
			DefWindowProc (hwnd, message, wParam, lParam);
		}
	}
\end{lstlisting}

Ze względu na jej użyteczność, stabilność oraz najlepsze wsparcie dla
DirectX w dalszej części projektu do obsługi okna użyte zostanie WinAPI. Rozważana była natomiast jeszcze jedna opcja opisana poniżej.

\subsection{UWP / Windows App SDK}

Skrót od \emph{Universal Windows Platform}. UWP zostało utworzone jako
zunifikowana platforma API aplikacji dla systemów Windows, konsol Xbox,
czy urządzeń mobilnych opartych na Windows Phone. Platforma została
opracowana z myślą o bezpieczeństwie poprzez użycia odizolowanego
środowiska, wydajności, wygodzie oraz wieloplatformości. Microsoft
stopniowo zaleca przejście ze standardu UWP do Windows App SDK o bardzo
podobnym API oraz architekturze. DirectX 11 oraz DirectX 12 posiadają oficjalne wsparcie dla tego
backendu \cite{ms:dxuwp:2024}, wraz z oficjalnymi szablonami projektów do programu
Visual Studio \cite{ms:dxuwpui:2024}. UWP posiada jednak wyraźne wady w postaci zamkniętej architektury oraz
problematycznej dystrybucji, gdyż do bezproblemowej instalacji wymagana
jest publikacja w Microsoft Store celem uzyskania podpisanej wersji
aplikacji \cite{ms:uwppackaging:2024}. Między innymi z tego względu po testach do projektu użyte zostało WinAPI
zamiast standardu UWP.

\section{Direct3D i język programowania}

Kolejną podstawową decyzją jest wybór wersji Direct3D oraz idącego za
nim języka programowania. Wybór ten będzie definiować dalszy wybór
bibliotek pomocniczych, architektury modułu oraz możliwości
implementacji bardziej złożonych funkcjonalności. Jako podstawę do testów w tej kategorii wybrany został prosty program
wyświetlający trójkąt równoboczny na ekranie. Każdy wierzchołek figury
zostaje pokolorowany innym kolorem, a wyświetlana jest ona na niebieskim
tle. Porównanie zostało wykonane na tle wydajności, skomplikowania
wymaganego kodu oraz potencjalnych możliwości rozwoju.

\subsection{Direct3D 9}

% Usunięto \begin{quote} i \end{quote}
Wydany w 2002r. był jednym z najbardziej popularnych wyborów wśród
twórców gier aż do wydania DirectX 11. Dzięki swojej prostocie użycia
oraz wsparciu szerokiej gamy systemów operacyjnych od Windows 98 wzwyż
\cite{falconfly:dxredist:2024} pozwalał na dotarcie do dużej grupy odbiorców niewielkim
nakładem pracy. Pomagał w tym także fakt, że wydanie to było bazowym API
graficznym konsoli Xbox 360 \cite{ms:xbox360bp:2024}. Preferowanym językiem obsługi DirectX 9 jest C++ i to w tym języku
implementowany był program testowy. Program testowy został rozbity na 2 części -- inicjalizację wykonywaną
raz oraz rysowanie wywoływane co klatkę obrazu. Taka organizacja pozwala
na lepsze rozdzielenie poszczególnych fragmentów oraz uniknięcie
problemów związanych z wyciekami pamięci. Poniżej przedstawione zostały
opisane pseudokodem fragmenty kodu, potrzebnego do implementacji
programu testowego.
% Usunięto \end{quote}

\subsubsection{Inicjalizacja}
\begin{lstlisting}[caption={Pseudokod inicjalizacji Direct3D 9 (oryginalna treść)}, label={lst:d3d9_init}]
	void init() {
		// Własna funkcja tworząca okno przy pomocy WinAPI
		CreateWindow();
		d3d = Direct3DCreate9(); // Utworzenie uchwytu interfejsu
		
		// Utworzenie uchwytu do urządzenia GPU
		dev = d3d->CreateDevice();
		
		// Utworzenie bufora wierzchołków, zawierającego informacje o położeniu 
		// i kolorach poszczególnych Vertex'ów
		verticesData = CreateVerticesData();
		vBuff = dev->CreateVertexBuffer(verticesData);
		memcpy(); // Przekopiowanie danych wierzchołków do bufora
	}
\end{lstlisting}

\subsubsection{Rysowanie}
\begin{lstlisting}[caption={Pseudokod rysowania Direct3D 9 (oryginalna treść)}, label={lst:d3d9_render}]
	void render() {
		dev->Clear(); // Czyszczenie bufora ekranu
		dev->BeginScene(); // Rozpoczęcie rysowania sceny
		
		// Ustawienie macierzy jednostkowej jako macierzy transformacji sceny		
		dev->SetTransform(VIEW, identityMatrix);
		
		// Ustawienie macierzy projekcji
		dev->SetTransform(PROJECTION, projectionMatrix);
		dev->SetFVF(); // Ustawienie formatu wierzchołków
		
		// Ustawienie bufora wierzchołków jako źródła danych.
		dev->SetStreamSource(vBuff);
		
		// Ustawienie typu wielokątów -- trójkątów		
		dev->DrawPrimitive(TRIANGLE_LIST);	
		dev->EndScene(); // Zakończenie rysowania sceny		
		dev->Present(); // Wyświetlenie wyrysowanego obrazu
	}
\end{lstlisting}

Na opisanych fragmentach kodu widoczna jest wspomniana już wcześniej
prostota obsługi tej wersji API. Większość operacji znajduje się za
warstwą abstrakcji po stronie sterownika karty graficznej. Pozwala to na
łatwiejsze pisanie kodu, ale daje programistom bardzo mało kontroli nad
procesem rysowania. Częściowym rozwiązaniem tego problemu miał być
wydany jako następca dziewiątki DirectX 11.

\subsection{Managed DirectX}

% Usunięto \begin{quote} i \end{quote}
Managed DirectX jest wydaną w 2002r. biblioteką, mającą na celu
przeniesienie API DirectX 9 do świata języka C\# i framework'u .NET,
upraszczając w ten sposób proces pisania niskopoziomowych aplikacji
akcelerowanych sprzętowo. Wydana została jedynie wersja pierwsza z planowaną drugą, która została
wycofana na rzecz Microsoft XNA i nigdy nie została upubliczniona. Ze
względu na to MDX posiada obsługę jedynie dla DirectX w wersji 9 oraz
architektury x86, bez wsparcia dla x64. Przykładowy pseudokod przedstawiający implementację testowej aplikacji
został przedstawiony poniżej:
% Usunięto \end{quote}

\subsubsection{Inicjalizacja}
% Zmieniono styl na csharpstyle, zachowując oryginalną treść
\begin{lstlisting}
	
	void init() {
		CreateWindow();
		
		// Utworzenie uchwytu do urządzenia
		dev = new Deivce();
		
		// Utworzenie i uzupełnienie bufora wierzchołków
		vBuffer = new VertexBuffer();
		
		FillVertexBuffer();
	}
	
\end{lstlisting}

\subsubsection{Rysowanie}
% Zmieniono styl na csharpstyle, zachowując oryginalną treść
\begin{lstlisting}
	
	void render() {
		// Czyszczenie ekranu
		dev->Clear();
		
		// Rozpoczęcie rysowania
		dev->BeginScene();
		
		// Ustawienie formatu wejściowego wierzchołków
		dev->VertexFormat = CustomFormat;
		
		// Rysowanie wielokątów
		dev->DrawPrimitives(TRIANGLE_LIST, 0, 1);
		
		// Zakończenie rysowania
		dev->EndScene();
		
		// Wyświetlenie sceny
		dev->Present();
	}
	
\end{lstlisting}

\subsection{Direct3D 11}

Niebezpośredni następca DirectX 9. Pomiędzy nimi wydana została wersja
10, ale nie zdobyła ona nadmiernej popularności i była jedynie iteracją,
przygotowującą na wydanie jedenaste, które stało się de facto podstawą
dla większości gier i programów wydanych w trakcie ósmej generacji
konsol do gier. DirectX 11 jest także podstawowym API graficznym systemu
Windows 8 / 8.1 oraz konsoli Xbox One \cite{ms:raisingbar:2024}. Pod względem architektury to wydanie jest krokiem w stronę
niskopoziomowej obsługi akceleracji sprzętowej, oddając deweloperom do
dyspozycji wyraźnie większą kontrolę nad sprzętem i jego
funkcjonowaniem. Zmiany obejmują między innymi wsparcie wielowątkowości,
programowalny pipeline graficzny z obsługą teselacji i compute shaders,
czy bardziej zaawansowany model shader'ów w wydaniu Shader Model 5.0. Niestety odbywa się to kosztem złożoności obsługi nowego API, między
innymi poprzez ręczne zarządzenie niektórymi elementami rysowania,
takimi jak Swapchain czy Assembler Input Layout. Tak jak w poprzednim przypadku, uproszczony pseudokod podzielony został
na 2 główne funkcje -- inicjalizacja i rysowanie. Wykorzystany został
tryb rysowania Immediate do uproszczenia przykładu.

\subsubsection{Inicjalizacja}
\begin{lstlisting}[caption={Pseudokod inicjalizacji Direct3D 11 (oryginalna treść)}, label={lst:d3d11_init}]
	
	void init() {
		CreateWindow();
		
		// Utworzenie uchwytu do wirtualnego urządzenia
		dev = D3D11CreateDevice();
		
		// Utworzenie uchwytu do fizycznego urządzenia GPU
		adapter = dev->GetAdapter();
		
		// Utworzenie swapchain -- obiektu odpowiedzialnego za przechowywanie}
		// klatek wygenerowanych i w trakcie rysowania
		swapchain = CreateSwapChainForHwnd();
		
		// Utworzenie obiektu RTV (Render Target View).
		rtv = dev->CreateRenderTargetView();
		
		// Utworzenie bufora głębi, stencil oraz widoku
		CreateDepthStencilView();
		
		// Utworzenie i skompilowanie shader'ów.
		vShader = dev->CreateVertexShader();
		pShader = dev->CreatePixelShader();
		
		// Utworzenie układu danych wejściowych
		CreateInputLayout();
		
		// Utworzenie buforów danych wejściowych
		CreateInputBuffers();
	}
	
\end{lstlisting}

\subsubsection{Rysowanie}
\begin{lstlisting}[caption={Pseudokod rysowania Direct3D 11 (oryginalna treść)}, label={lst:d3d11_render}]
	
	void render() {
		// Wyczyszczenie bufora rysowania
		rtv->Clear();
		
		// Ustawienie macierzy projekcji, widoku i świata.
		context->SetWorldViewProjectionMatrix();
		
		// Ustawienie oraz skonfigurowanie shaderów używanych w procesie
		// rysowania
		context->SetVertexShader(vShader);	
		context->SetConstantBuffers();
		context->SetFragmentShader(pShader);
		
		// Rysowanie i prezentowanie wygenerowanej klatki obrazu
		context->Draw();
		swapchain->Present();
	}
	
\end{lstlisting}

Opisany pseudokod nie oddaje w pełni złożoności oraz kontroli jaką
oddaje w ręce programisty DirectX 11. Program wykonujący tą samą funkcję
podwoił swoją objętość względem wersji 9, ale pozwala to na znacznie
większą kontrolę nad pełnym procesem rysowania. Jest to nadal tylko
ułamek kontroli, jaką oddaje jego następca.

\subsection{Direct3D 12}

Bezpośredni następca DirectX 11 całkowicie zmieniający sposób
postrzegania i użytkowania API graficznego. Direct3D 12 oddaje w ręce
programisty kontrolę nad większością części procesu rysowania grafiki
3D. Wadą takiego podejścia jest zdecydowane zwiększenie nakładu pracy
wymaganego do prawie każdej operacji. Szczególny nacisk po stronie
programisty został położony na zarządzenie pamięcią, w tym zarządzanie
używanymi zasobami i zwalnianie ich w odpowiednim czasie. Zwalnia to
sterownik z konieczności pełnienia tej roli, co przekłada się na
wyraźnie zwiększoną wydajność, ale może prowadzić do wycieków pamięci i
bardziej złożonej architektury programu. Jak w poprzednich tak i w tym przypadku przykładowa aplikacja została
podzielona na inicjalizację oraz rysowanie, opisane pseudokodem poniżej.

\subsubsection{Inicjalizacja}
\begin{lstlisting}[caption={Pseudokod inicjalizacji Direct3D 12 (oryginalna treść)}, label={lst:d3d12_init}]
	
	void init() {
		CreateWindow();
		
		// Stworzenie "fabryki" obiektów DXGI
		factory = CreateDXGIFactory();
		
		// Stworzenie uchwytu do wirtualnego urządzenia
		dev = D3D12CreateDevice();
		
		// Stworzenie kolejki komend. Przy jej pomocy wysyłane będą polecenia do GPU.
		queue = dev- $>$CreateCommandQueue();
		swapchain = factory->CreateSwapChainForHwnd();
		
		// Utworzenie sterty deskryptora. Będzie ona zawierać wejściowe
		descHeap = dev->CreateDescriptorHeap();
		
		// Tworzymy odpowiednią ilość widoków RTV. Każdy z osobna
		// dla każdej klatki obrazu "w locie"
		for(int i = 0; i < FrameCount; i++) {
			rtv{[}i{]} = dev->CreateRenderTargetView();
		}
		
		// Utworzenie alokatora dla listy komend
		allocator = dev->CreateCommandAllocator();
	
		// Utworzenie sygnatury root'a. 
		// Będzie zawierać opis formatu danych wejściowych
		rootSignature = dev->CreateRootSignature();
		
		// Utworzenie i kompilacja shader'ów
		vShader = CreateVertexShader();
		pShader = CreatePixelShader();
		
		// Utworzenie stanu "linii produkcyjnej" rysowania grafiki
		pipeline = dev->CreateGraphicsPipelineState();
		
		// Utworzenie listy komend wywoływanej na kolejce komend
		cList = dev->CreateCommandList();
		
		// Utworzenie zasobów bufora wierzchołków oraz uzupełnienie go
		vBuffer = dev->CreateCommitedResource();	
		memcpy(vBuffer);
		
		// Utworzenie "płotka", będącego strukturą synchronizacji między GPU, a CPU
		syncFence = dev->CreateFence();
	}
	
\end{lstlisting}

\subsubsection{Rysowanie}
\begin{lstlisting}[caption={Pseudokod rysowania Direct3D 12 (oryginalna treść)}, label={lst:d3d12_render}]
	
	void render() {
		// Uzupełnienie listy komend
		allocator->Reset(); // Resetowanie alokatora oraz listy komend
		cList->Reset(); // Resetowanie listy komend
		
		// Ustawienie bazowej sygnatury opisującej format danych wejściowych	
		cList->SetRootSignature(rootSignature);
		
		// Stworzenie bariery, która jest elementem synchronizacji 
		// i zmienia stan obiektu powierzchni rysowania na RENDER_TARGET
		cList->ResourceBarrier(RESOURCE_STATE, RENDER_TARGET);
		
		// Wyczyszczenie powierzchni rysowania
		cList->ClearRenderTargetView();
		
		// Polecenie rysowania
		cList->DrawInstanced();
		
		// Ponowna bariera przywracająca stan powierzchni rysowania
		cList->ResourceBarrier(RENDER_TARGET, RESOURCE_STATE);
		
		// Wysłanie listy komend i wyświetlenie klatki obrazu
		queue->ExecuteCommandLists(cList);
		swapchain->Present();
		
		// Synchronizacja poprzez oczekiwanie na ukończenie pracy GPU.
		// Jest to wersja uproszczona. 
		// Poprawnym jest utworzenie kilku klatek "w locie"
		// i oczekiwanie na nie w odpowiedniej kolejności, a nie tylko na poprzednią
		newValue = queue->Signal(syncFence);
		
		WaitForSingleObject(syncFenceEvent);
	}
	
\end{lstlisting}

API Direct3D 12 jest biblioteką bardziej zaawansowaną niż jest to
wymagane dla tego projektu, szczególnie uwzględniając prowadzenie go w
pojedynkę oraz złożoność wersji dwunastej. W związku z tym wykorzystane
zostanie API Direct3D 11, będące wypośrodkowaniem między możliwościami i
nowoczesnością, a złożonością dla programisty. Z tego względu w dalszych
przykładach oraz finalnym projekcie użyty zostanie język C++ z IDE
Microsoft Visual Studio oraz kompilatorem Microsoft Visual C++.

\section{Wczytywanie tekstur}

\subsection{stb\_image}

Element pakietu STB, będącego zbiorem otwarto źródłowych bibliotek
pomocniczych, zawierającym kod między innymi do obsługi wczytywania
obrazów (stb\_image), generowania szumu (stb\_perlin), wczytywania
czcionek (stb\_truetype), czy wykrywania wycieków pamięci
(stb\_leakcheck) \cite{github:stb:2024}. Biblioteki te charakteryzują się prostym API,
łatwym użyciem i brakiem kosztów licencyjnych. Stb\_image pozwala na proste wczytywanie i dekodowanie wielu popularnych
formatów obrazów, między innymi JPG, PNG, TGA, czy BMP \cite{github:stb:2024}. Jego
działanie nie jest uzależnione od żadnego API graficznego czy biblioteki
UI, co pozwala na uniwersalne użycie do własnego modułu. Poniżej przedstawiony został przykład użycia biblioteki stb\_image do
wczytania pliku cat.png w formacie ARGB.

\begin{lstlisting}[caption={Przykład użycia stb\_image (oryginalna treść)}, label={lst:stb_image}]
	
	#include <stb_image.h>
	
	int main() {
		int imgWidth;
		int imgHeight;
		int imgChannels;
		
		// Wczytanie pliku z użyciem funkcji stbi_load()
		unsigned char *imgData = stbi_load(
				"cat.png", 
				&imgWidth, 
				&imgHeight, 
				&imgChannels, 
				4
		);
		
		processImage(imgData); // Użycie wczytanego obrazu
		free(imgData); // Zwolnienie pamięci
	}
	
\end{lstlisting}

\subsection{DirectXTK}

Skrót od nazwy „\emph{DirectX Tool Kit}'' - stworzony przez Chuck'a
Walbourn'a pod kontrolą firmy Microsoft zestaw bibliotek,
zaprojektowanych jako pomoc przy pisaniu programów wykorzystujących API
DirectX 11 \cite{github:directxtk:2024}. Zaprojektowany jako następca wycofanej biblioteki
D3DX \cite{walbourn:directxtk:2024}. Pakiet został później przeportowany do użycia z DirectX
12 \cite{github:directxtk12:2024}. Zawiera między innymi moduły odpowiedzialne za rysowanie
sprite'ów 2D (SpriteBatch), rysowanie tekstu 2D (SpriteFont),
wczytywanie modeli 3D (Model), czy rozszerzenie o często używane typy
wierzchołków (VertexTypes) \cite{walbourn:directxtk:2024}.

Komponent WICTextureLoader odpowiedzialny jest za wczytywanie tekstur
przy pomocy API Windows Imaging Component, w skrócie \emph{WIC}. Obsługuje zdecydowaną większość formatów graficznych, w tym także format
DDS, którego obsługi brakuje w przypadku stb\_image. Wczytanie tekstury przy pomocy DirectXTK opisane jest przy okazji bardzo
dobrej dokumentacji projektu \cite{github:directxtk:sprites:2024}.

\begin{lstlisting}[caption={Przykład użycia DirectXTK WICTextureLoader (oryginalna treść)}, label={lst:directxtk_wic}]

	#include <WICTextureLoader.h>
	
	// Utworzenie nowego zasobu
	ComPtr<ID3D11Resource> resource;
	
	const wchar_t* path = L"cat.png";
	
	int main() {
		// Wczytanie przy pomocy funkcji z DirectXTK
		CreateWICTextureFromFile(
				device, 
				path, 
				resource.GetAddressOf(),
				m_texture.ReleaseAndGetAddressOf()
		);
		
		// Konwersja typu na typ tekstury Direct3D
		ComPtr<ID3D11Texture2D> cat;
		DX::ThrowIfFailed(resource.As(&cat));
	}
	
\end{lstlisting}

Ze względu na świetną integrację z systemem DirectX jako bibliotekę
obsługi tekstur w projekcie wybrany został DirectXTK.

\section{Wczytywanie modeli 3D}

Kolejnym po teksturach formatem często wykorzystywanym w silnikach 3D są
trójwymiarowe modele obiektów. Mogą one przyjmować postać plików o
formatach takich jak FBX, CMO, DAE, OBJ, czy glTF.

\subsection{DirectXTK}

Wspomniana już przy okazji wczytywania tekstur biblioteka pomocnicza
DirectX Tool Kit zawiera także komponent odpowiedzialny za wczytywanie
modeli. Jego użycie jest proste i intuicyjne, do tego dobrze pasowałoby
do użycia razem z elementem tekstur. Przykład użycia przedstawiony
został poniżej.

\begin{lstlisting}[caption={Przykład użycia DirectXTK Model (oryginalna treść)}, label={lst:directxtk_model}]
	
	// Utworzenie "fabryki" efektów, wymaganej do użycia funkcji
	m_fxFactory = std::make_unique<EffectFactory>(device);
	
	// Wczytanie modelu przy użyciu funkcji CreateFromCMO
	auto teapot = Model::CreateFromCMO(device, L"teapot.cmo", *m_fxFactory);
	
	// Funkcja pomocnicza rysowania
	teapot->Draw(...);
	
\end{lstlisting}

Użycie DirectXTK posiada jednak znaczącą wadę, a jest nią ograniczenie
do formatów CMO oraz SDKMESH \cite{github:directxtk:model:2024}. Brak wsparcia dla popularnych
typów FBX, OBJ, czy DAE.

\subsection{Assimp}

W pełnej nazwie \emph{Open Asset Import Library}, jest otwarto źródłową
biblioteką z funkcjonalnością wczytywania modeli z ponad 40 różnych
formatów plików 3D \cite{github:assimp:2024}. Wśród narzędzi znajdują się także funkcje
postprocessing'u modeli, takie jak wyliczanie wartości normalnych,
triangulacji, czy optymalizacji wierzchołków.

Użycie assimp jest bardziej złożone od DirectXTK, ale nadal dobrze
opisane w licznych poradnikach oraz dokumentacji. Poniżej przedstawiony
został uproszczony przykład wczytania przykładowego pliku teapot.fbx,
który został sporządzony na podstawie pełnego dema dostępnego w
dokumentacji projektu Assimp \cite{github:assimp:dx11sample:2024}.

\begin{lstlisting}[caption={Przykład użycia Assimp (oryginalna treść)}, label={lst:assimp_load}]

	void load() {
		Assimp::Importer importer;
		const aiScene* pScene = importer.ReadFile("teapot.fbx", ...);
		processNode(pScene->mRootNode, pScene);
	}
	
	void processNode(aiNode* node, const aiScene* scene) {
		// Wczytanie modeli node'a z tablicy
		// Dla uproszczenia pominięto pętle po mesh'ach
		processMeshes(scene->mMeshes);
		
		// Rekursywne wczytanie gałęzi
		for(uint i = 0; i < node->mNumChildren; i++) {
			processNode(node->mChildren{[}i{]}, scene);		
		}
	}
	
	void processMesh(aiMesh* mesh, const aiScene* scene) {		
		// Dla uproszczenia wczytywanie zostało przedstawione
		// przy pomocy pseudofunkcji loadFrom o definicji
		// loadFrom(data, n);
		
		vertices = loadFrom(mesh->mVertices, mesh->mNumVertices);
		UVs = loadFrom(mesh->textcoord, mesh->mNumVertices);
		faces = loadFrom(mesh->mFaces, mesh->mNumFaces);
		
		indices = loadFrom(mesh->mIndices, mesh->mNumFaces * face.mNumIndices);
		materials = scene->mMaterials{[}mesh->mMaterialIndex{]};
	}
	
\end{lstlisting}

Ze względu na wsparcie większej ilości formatów oraz wyraźnie lepszą
dokumentację jak i wsparcie społeczności, do projektu użyta zostanie
biblioteka Assimp.

\section{Biblioteka pomocnicza operacji matematycznych}

Do wydajnego rysowania trójwymiarowego potrzebne jest wykonywanie dużej
ilości złożonych operacji matematycznych, w szczególności operacji
macierzowych oraz wektorowych. Większość z nich odbywa się po stronie
API oraz karty graficznej i programista nie musi się nad nimi
zastanawiać, lecz część z tych operacji musi być wykonywania także po
stronie kodu przez niego pisanego. W celu uproszczenia oraz abstrakcji
takich operacji powstało kilka bibliotek ułatwiających to zadanie.

\subsection{DirectXMath}

Podstawowa biblioteka pomocnicza używana przez większość projektów
opartych o DirectX. Wykorzystuje rozszerzenia typu SIMD (\emph{Single
Instruction Multiple Data}) do wydajnego wykonywania złożonych operacji
opartych o algebrę liniową. Biblioteka wspiera architektury x86, x64
oraz ARM/ARM64.

Dzięki swojej konstrukcji oraz założeniom DXMath oddaje programiście
bardzo dużą kontrolę nad swoim działaniem, co pozwala na lepsze
zrozumienie działania oraz wyraźną poprawę wydajności. Odbywa się to
niestety kosztem prostoty oraz czytelności kodu. Aby zaradzić na ten
problem powstał wrapper SimpleMath.

\subsection{SimpleMath}

Wrapper nad DirectXMath. SimpleMath będący elementem DirectXTK upraszcza
większość operacji DXMath i zwalnia programistę z konieczności myślenia
w paradygmacie SIMD. Upraszcza to pracę z kodem oraz poprawia jego
czytelność, kosztem niewielkiego kosztu wydajnościowego. Biblioteka
skonstruowana jest w taki sposób, aby nadal możliwe było korzystanie z
DirectXMath w krytycznych dla wydajności sekcjach kodu, więc większość
kosztów wydajnościowych może zostać wyraźnie zredukowanych, bez straty
na czytelności kodu \cite{github:directxtk:simplemath:2024}. Z tego względu SimpleMath został wybrany
jako wykorzystana w projekcie biblioteka obsługi równań liniowych.

\section{Podsumowanie}

Poniżej przedstawione zostało podsumowanie użytych w projekcie
bibliotek.

% \begin{longtable}
% \toprule

% \begin{minipage}
% 	\textbf{Kategoria}
% \end{minipage} & \begin{minipage}[b]{\linewidth}\raggedright
% 	\textbf{Wybrana biblioteka}
% \end{minipage} \\
% \midrule
% \endhead
% API systemu okien & WinAPI \\
% API graficzne & DirectX 11 \\
% Język programowania & C++ \\
% Wczytywanie tekstur & DirectXTK / WICTextureLoader \\
% Wczytywanie modeli & Assimp \\
% Biblioteka pomocnicza do równań matematycznych & DirectXTK / SimpleMath \\
% \bottomrule
% \end{longtable}

\begin{center}
	\begin{tabular}{ |l r| }
		\hline
		\textbf{Kategoria} & \textbf{Biblioteka} \\
		\hline
		API systemu okien & WinAPI \\
		API graficzne & Direct3D 11 \\
		Język programowania & C++ \\ 
		Wczytywanie tekstur & DirectXTK / WICTextureLoader \\
		Wczytywanie modeli & Assimp \\
		Biblioteka pomocnicza równań matematycznych & DirectXTK / SimpleMath \\
		\hline
	\end{tabular}
\end{center}