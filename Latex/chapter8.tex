\chapter{Testy - wydajność i ograniczenia}
	\begin{comment}
		Testy:
			- Porównanie dla trybu "Natywnego", "URM::Core", "URM::Engine".
			- Porównanie złożoności kodu.
			1) Wydajność przetwarzania wierzchołków.
				* Testy wyświetlania obiektów o bardzo złożonej topologii.
				- Oszacowanie maksymalnej ilości wierzchołków.
				- Sprawdzenie narzutu na CPU i GPU.
				+ Optymalizacje vertex shader'a.
				
			2) Wydajność generowania oświetlenia
				* Sprawdzenie narzutu wydajności wielu źródeł światła.
				- Oszacowanie maksymalnego limitu ilości świateł w scenie.
				+ Dodanie bardziej zaawansowanego silnika obliczania efektów oświetlenia.
				
			3) Testy wydajności pixel shader'a.
				- Sprawdzenia skalowania wydajności zależnie od rozdzielczości.
				- Sprawdzenie bottleneck'u (CPU / GPU).
				
			4) Instance rendering.
				- Sprawdzenie jak moduł radzi sobie z rysowaniem wielu instancji tego samego lub różnych obiektów.
				- Porównanie wydajności z różnymi shader'ami i tym samym shader'em.
				
			5) Wydajność wczytywania modeli i tekstur z pliku.
				- Porównanie różnych formatów.
				- Porównanie różnych opcji preprocessingu.
				- Porównanie pierwszego i kolejnych wczytań modelu (AssetManager).
				
			6) Testy wydajności macierzy transformacji w Transform.
				- Określenie złożoności obliczeniowej poszczególnych funkcji.
				- Oszacowanie maksymalnej głębokości oraz szerokości drzewa obiektów.
				- Sprawdzenie narzutu wielokrotnego ustalania pozycji.
				+ Cache'ing.
				
	\end{comment}