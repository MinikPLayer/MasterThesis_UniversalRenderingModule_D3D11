\chapter{Platforma testowa i konfiguracja}

Przed przystąpieniem do finalnych testów określić należy najpierw
platformę testową. Dzięki takiemu podejściu możliwe jest lepsze określenie zakresu
badań oraz łatwiejsze monitorowanie zmian w wydajności.

Wybraną podstawową platformą deweloperską jest laptop \textbf{Lenovo Legion Slim 5-14}, składający się z następujących komponentów:

\begin{center}
	\begin{tabular}{ |l r| }
		\hline
		\textbf{Kategoria} & \textbf{Komponent} \\
		\hline
		CPU & \textbf{AMD Ryzen 7 7840HS} \\
		GPU1 & \textbf{NVIDIA RTX 4060 8GB} \\
		GPU2 & \textbf{AMD Radeon 780M (zintegrowana)} \\
		RAM & \textbf{32GB LPDDR5x 6400 MT/s} \\
		SSD & \textbf{Samsung 980 Pro 2TB} \\
		Sterownik graficzny NVIDIA & \textbf{577.0} \\
		Sterownik graficzny AMD & \textbf{Adrenaline 24.10.34.01} \\
		System operacyjny & \textbf{Windows 11 Pro x64, wersja 24H2} \\
		\hline
	\end{tabular}
\end{center}

Testy na dwóch różnych układach graficznych o wyraźnie odbiegających od siebie osiągach pozwoliły na określenie gdzie leży ograniczenie danego testu - po stronie procesora, czy GPU.

Do wyeliminowania zmiennych termicznych testy przeprowadzone zostały ze zbalansowanym trybem energetycznym, ale z manualnie ustawioną prędkością wentylatorów na maksymalne obroty. W tym samym celu ustawione ręcznie zostały limity APU (połączonych CPU i GPU2) oraz GPU1 i prezentują się następująco:

\begin{center}
	\begin{tabular}{ |l r| }
		\hline
		\textbf{Parametr} & \textbf{Wartość} \\
		\hline
		APU - Peak & \textbf{55W} \\
		APU - Short term & \textbf{55W} \\
		APU - Long Term & \textbf{55W} \\
		APU - Cross load & \textbf{55W} \\
		APU - Temperature limit & \textbf{100°C} \\
		GPU1 - TDP & \textbf{90W} \\
		GPU1 - Dynamic Boost & \textbf{0W} \\
		GPU1 - Temperature limit & \textbf{87°C} \\
		\hline
	\end{tabular}
\end{center}

Do kompilacji programu użyty został system budowy MSBuild w wersji 17.14.8.25101 wraz z kompilatorem \textit{cl} o numerze edycji 19.44.35207.1. Do jego konfiguracji użyte zostały następujące parametry:

\begin{center}
	\begin{tabular}{ |l r| }
		\hline
		\textbf{Parametr} & \textbf{Wartość} \\
		\hline
		Optimization & \textbf{Maximum (O2)} \\
		Enable Intrinsic Functions & \textbf{Yes (/Oi)} \\
		Favor Size Or Speed & \textbf{Favor fast code (/Ot))} \\
		Whole Program Optimization & \textbf{Yes (/GL)} \\
		\hline
	\end{tabular}
\end{center}