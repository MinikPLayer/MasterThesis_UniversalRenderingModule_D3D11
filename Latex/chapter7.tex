\chapter{Przyjęta metoda rysowania sceny}
\label{ChapterSceneDrawingMethod}
	Ze względu na założenie uniwersalności - w tym pod względem platformy docelowej - jako podstawa i punkt wyjścia przyjęty został sposób rysowania \textit{Forward Rendering}, ze względu na lepszą skalowalność między systemami o różnej mocy obliczeniowej oraz możliwość użycia większości efektów graficznych. W związku z tym wyborem koszt obliczeniowy oraz pamięciowy rysowania sceny jest bazowo niski, lecz dodanie dużej ilości źródeł światła wiąże się ze zwiększonym narzutem obliczeniowym.
	
\section{Modyfikacja sposobu obliczania oświetlenia}
	Wyraźnym odejściem względem klasycznego \textit{Forward Rendering} jest sposób obliczania oświetlenia. Podstawową metodę można przedstawić w uproszczonej formie przy pomocy pseudokodu:
	
	\begin{lstlisting}[label={lst:shaderForwardLightCalculations}]
UNIFORM Light[] lights; // Przekazane do shader'a dane o źródłach światła,
UNIFORM int numLights;  // oraz o ilości aktywnych źródeł w tablicy.
		
Color resultColor = BLACK;
// Pętla sumująca obliczone wartości oświetlenia
for (i = 0; i < numLights; i += 1) {
    resultColor += CalculateLight(lights[i]);
}

return resultColor;
	\end{lstlisting}
	
	Takie podejście ma jednak dwa ograniczenia - wydajność oraz ograniczenie maksymalnej ilości źródeł światła w scenie.	Pierwsze wynika ze sposobu działania współczesnych układów graficznych oraz kompilatorów. Dokładniej ograniczenie wydajności wynika z tzw. \textit{Register spilling}, czyli sytuacji w której na układzie graficznym brakuje rejestrów do pełnego pokrycia zmiennych, przez co koniecznym jest korzystanie z powolnej pamięci globalnej \cite{amd:gpuopen:RegisterSpilling}. Możliwym jest też spadek wydajności przez zmniejszenie skuteczności mechanizmu \textit{rozwijania pętli} kompilatora, przez znaczne zwiększenie maksymalnej jej długości. Drugie ograniczenie spowodowane jest limitem ilości elementów bufora stałych w API Direct3D - 4096 \cite{microsoft:Direct3D11:ResourceLimits}, co ogranicza ilość źródeł światła do około 1400. 
	
	Oba problemy zostały rozwiązane dzieląc pełną listę świateł na fragmenty o stałej wielkości (na przykład 64). Następnie zamiast wywoływać proces renderowania dla wszystkich źródeł, uruchamiany jest po kolei dla kolejnych segmentów, a wynik dodawany jest do bufora kolorów przy pomocy mechanizmu \textit{Blending}, opartego o klasę z segmentu \textit{D3DBlendState}. Aby takie podejście działało poprawnie pierwsze wywołanie procesu rysującego odbywa się przy wyłączonym blendingu.