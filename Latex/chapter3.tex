\chapter{Platforma testowa i konfiguracja}

Przed przystąpieniem do finalnych testów określić należy najpierw
platformę testową. Dzięki takiemu podejściu możliwe jest lepsze określenie zakresu
badań oraz łatwiejsze monitorowanie zmian w wydajności.

Wybraną podstawową platformą deweloperską jest laptop \textbf{Lenovo Legion Slim 5-14}, składający się z następujących komponentów:

\begin{itemize}
	\item \textbf{CPU}: AMD Ryzen 7 7840HS @ 55W
	\item \textbf{GPU 1}: NVIDIA RTX 4060 8GB @ 90W
	\item \textbf{GPU 2}: AMD Radeon 780M (zintegrowana)
	\item \textbf{RAM}: 32GB LPDDR5x, 6400 MT/s
	\item \textbf{SSD}: Samsung 980 Pro 2TB
	\item \textbf{Sterownik graficzny NVIDIA}: 565.90
	\item \textbf{System operacyjny}: Windows 11 Pro x64, wersja 24H2
\end{itemize}

Do wyeliminowania zmiennych termicznych testy przeprowadzone zostały ze zbalansowanym trybem energetycznym, ale z manualnie ustawioną prędkością wentylatorów na maksymalne obroty.

Ustawione ręcznie limity podzespołów prezentują się następująco:

\begin{center}
	\begin{tabular}{ |l r| }
		\hline
		\textbf{Parametr} & \textbf{Wartość} \\
		\hline
		CPU - Peak & \textbf{55W} \\
		CPU - Short term & \textbf{55W} \\
		CPU - Long Term & \textbf{55W} \\
		CPU - Cross load & \textbf{55W} \\
		CPU - Temperature limit & \textbf{100°C} \\
		GPU - TDP & \textbf{90W} \\
		GPU - Dynamic Boost & \textbf{0W} \\
		GPU - Temperature limit & \textbf{87°C} \\
		\hline
	\end{tabular}
\end{center}

Do kompilacji programu użyty został system budowy MSBuild w wersji 17.14.8.25101 wraz z kompilatorem \textit{cl} o numerze edycji 19.44.35207.1. Do jego konfiguracji użyte zostały następujące parametry:

\begin{center}
	\begin{tabular}{ |l r| }
		\hline
		\textbf{Parametr} & \textbf{Wartość} \\
		\hline
		Optimization & \textbf{Maximum (O2)} \\
		Enable Intrinsic Functions & \textbf{Yes (/Oi)} \\
		Favor Size Or Speed & \textbf{Favor fast code (/Ot))} \\
		Whole Program Optimization & \textbf{Yes (/GL)} \\
		\hline
	\end{tabular}
\end{center}