\chapter{Platforma testowa}

Przed przystąpieniem do prac nad projektem określić należy najpierw
platformę testową, na której i dla której budowany oraz testowany jest
system. Dzięki takiemu podejściu możliwe jest lepsze określenie zakresu
badań oraz łatwiejsze monitorowanie zmian w wydajności.

Wybraną podstawową platformą deweloperską jest komputer składający się z
następujących komponentów:

\begin{itemize}
	\item \textbf{CPU}: AMD Ryzen 9 7900X @ 140W
	\item \textbf{GPU}: NVIDIA RTX 4070 12GB
	\item \textbf{RAM}: 64GB DDR5, 6000 MT/s
	\item \textbf{SSD}: Samsung 980 Pro 2TB
	\item \textbf{Sterownik graficzny NVIDIA}: 565.90
	\item \textbf{Wersja systemu operacyjnego}: Windows 11 Pro, x64, wersja 24H2
\end{itemize}

Dodatkowo do weryfikacji wyników użyty został także laptop testowy \textbf{ASUS Vivobook S16 2025}, którego specyfikacja
prezentuje się następująco:

\begin{itemize}
	\item \textbf{CPU}: AMD Ryzen AI 9 HX 370
	\item \textbf{GPU1}: AMD Radeon 890M
	\item \textbf{RAM}: 32GB LPDDR5x, 7500 MT/s
	\item \textbf{SSD}: Samsung 980 Pro 2TB
	\item \textbf{Sterownik graficzny AMD}: 24.30.22.03
	\item \textbf{Wersja systemu operacyjnego}: Windows 11 Pro, x64, wersja 24H2
\end{itemize}