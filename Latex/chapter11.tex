\chapter{Podsumowanie}
Dziedzina grafiki trójwymiarowej rozwija się coraz szybciej. Wszędzie można usłyszeć o nowe technikach renderowania, układach graficznych, silnikach do gier, skrótem - kolejnej rewolucji. W tak szybko pędzącym świecie warto znać podstawy, na których wszystkie te nowinki technologiczne się opierają. Przydatnym jest też mieć taki punkt wyjścia, z którego możliwym jest rozwijanie bardziej zaawansowanych technik i projektów. Z takim właśnie założeniem powstał opisywany tu projekt. 

Stworzony przede wszystkim z myślą o byciu dobrą podstawą do dalszego rozwoju, tak aby żadna dodatkowa funkcjonalność nie blokowała w żaden sposób drogi do zewnętrznego rozszerzania, nawet jeśli dane ulepszenie nie istniało w pierwotnym założeniu. Pozwala na to budowa warstwowa, założenia modularności i brak barier w dostępie do niższych poziomów implementacji. 

Nawet w przypadku braku chęci lub potrzeby większego rozwoju, zaimplementowane funkcjonalności - system materiałów, wczytywania modeli i tekstur z dysku, hierarchiczny układ sceny - są wystarczające do dużej części prostszych zastosowań. 

Na podstawie testów wydajność modułu jest wystarczająca do większości zadań, a ograniczenia są dobrze znane i ciężkie do osiągnięcia w realnych zastosowaniach, do których moduł został stworzony.  

Jeśli mimo wszystko okazałoby się to być za mało, moduł został opublikowany na otwartej licencji open source \cite{GitHub:Minik:MasterThesisUniversalRenderingModuleD3D11}, a kod napisany jest w taki sposób, aby możliwe było jego łatwe rozwijanie i poprawianie. Dzięki temu może być on podstawą lub wzorcem dla każdego, kto takiego projektu potrzebuje, a \textit{URM} ma możliwość stać się bardziej uniwersalnym niż większość innych, zamkniętych modułów i bibliotek.